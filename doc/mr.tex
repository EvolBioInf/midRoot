\nwfilename{}\nwbegindocs{0}\nwenddocs{}\nwbegindocs{1}\nwdocspar% ===> this file was generated automatically by noweave --- better not edit it
\section{Introduction}
A phylogeny summarizes the evolutionary relationships between a sample
of organisms. For example, the imaginary phylogeny in
Figure~\ref{fig:phy}A the genealogy of six organisms, also called
\emph{taxa}. The branches have lengths proportional to a measure of
evolutionary change, often the number of mutations per site. The
phylogeny is drawn in radial layout to emphasize that it has no
root. In other words, the most recent common ancestor of the taxa is
unknown. This is a standard result of many algorithms for
reconstructing phylogenies, for example the neighbor joining
algorithm~\cite{sai87:nei}. Still, in reality there was a common
ancestor. A popular method for placing it on the tree is called
midpoint rooting, where the root is located midpoint between the most
distant taxa. In the example phylogeny Figure~\ref{fig:phy}A, the most distant pair of taxa
is $t_1$/$t_4$. Rooting the phylogeny midpoint between them
returns Figure~\ref{fig:phy}B, where the branch leading to $t_1$ was
broken into a branch leading from the root to $t_1$ and a shorter branch
leading from the root to the rest of the tree.

\begin{figure}[ht]
  \begin{center}
    \begin{tabular}{ccc}
        \textbf{A} & \textcolor{white}{xxx}& \textbf{B}\\
        \begin{pspicture}(-0.931,-0.479)(3.000,3.798)
\psline(2.323647,3.497838)(3.000000,3.497838)
\rput(2.661823,3.697838){0.01}
\rput(0.547309,-0.282436){\rnode{n1}{}}\uput{4pt}[-27.295780]{-27.295780}(n1){$t_4$}
\rput(3.000000,2.997838){\rnode{n3}{}}\uput{4pt}[32.704224]{32.704224}(n3){$t_1$}
\rput(-0.765803,2.290000){\rnode{n5}{}}\uput{4pt}[82.704231]{272.704224}(n5){$t_3$}
\rput(-0.930999,2.234326){\rnode{n7}{}}\uput{4pt}[142.704239]{332.704224}(n7){$t_5$}
\rput(-0.758981,2.145557){\rnode{n6}{}}
\rput(-0.049123,1.040019){\rnode{n4}{}}
\rput(-0.243139,-0.384953){\rnode{n8}{}}\uput{4pt}[202.704224]{392.704224}(n8){$t_2$}
\rput(-0.170127,-0.479448){\rnode{n10}{}}\uput{4pt}[272.704193]{272.704193}(n10){$t_6$}
\rput(-0.176608,-0.342233){\rnode{n9}{}}
\rput(0.000000,0.000000){\rnode{n2}{}}
\ncline{n1}{n2}
\ncline{n3}{n4}
\ncline{n5}{n6}
\ncline{n7}{n6}
\ncline{n6}{n4}
\ncline{n4}{n2}
\ncline{n8}{n9}
\ncline{n10}{n9}
\ncline{n9}{n2}
\end{pspicture}
 & & \begin{pspicture}(0.000,0.000)(2.700,3.800)
\psline(2.008402,3.500000)(2.700000,3.500000)
\rput(2.354201,3.700000){0.01}
\rput(2.700000,0.000000){\rnode{n1}{}}\uput{4pt}[0.000000]{0.000000}(n1){$t_1$}
\rput(2.497362,0.600000){\rnode{n3}{}}\uput{4pt}[0.000000]{0.000000}(n3){$t_3$}
\rput(2.547157,1.200000){\rnode{n5}{}}\uput{4pt}[0.000000]{0.000000}(n5){$t_5$}
\rput(2.349360,0.900000){\rnode{n4}{}}
\rput(2.544390,1.800000){\rnode{n7}{}}\uput{4pt}[0.000000]{0.000000}(n7){$t_2$}
\rput(2.603868,2.400000){\rnode{n9}{}}\uput{4pt}[0.000000]{0.000000}(n9){$t_6$}
\rput(2.463474,2.100000){\rnode{n8}{}}
\rput(2.699308,3.000000){\rnode{n11}{}}\uput{4pt}[0.000000]{0.000000}(n11){$t_4$}
\rput(2.069954,2.550000){\rnode{n10}{}}
\rput(1.005584,1.725000){\rnode{n6}{}}
\rput(0.000000,0.862500){\rnode{n2}{}}
\ncangle[angleB=180,angleA=-90,armB=0]{n2}{n1}
\ncangle[angleB=180,angleA=-90,armB=0]{n4}{n3}
\ncangle[angleB=180,angleA=90,armB=0]{n4}{n5}
\ncangle[angleB=180,angleA=-90,armB=0]{n6}{n4}
\ncangle[angleB=180,angleA=-90,armB=0]{n8}{n7}
\ncangle[angleB=180,angleA=90,armB=0]{n8}{n9}
\ncangle[angleB=180,angleA=-90,armB=0]{n10}{n8}
\ncangle[angleB=180,angleA=90,armB=0]{n10}{n11}
\ncangle[angleB=180,angleA=90,armB=0]{n6}{n10}
\ncangle[angleB=180,angleA=90,armB=0]{n2}{n6}
\end{pspicture}

    \end{tabular}
  \end{center}
  \caption{An example phylogeny unrooted (\textbf{A}) and rooted
    midpoint between $t_1$ and $t_4$ shown in bold (\textbf{B}).}\label{fig:phy}
\end{figure}

\section{Implementation}
The program \texttt{midRoot} reads one or more trees and prints their
midpoint rooted versions. Like many C programs, it starts with include
statements, followed by function declarations, their implementations,
and finally the main function.

\nwenddocs{}\nwbegincode{2}\sublabel{NW0-5D8f4-1}\nwmargintag{{\nwtagstyle{}\subpageref{NW0-5D8f4-1}}}\moddef{midRoot.c~{\nwtagstyle{}\subpageref{NW0-5D8f4-1}}}\endmoddef\nwstartdeflinemarkup\nwenddeflinemarkup
\LA{}Include headers~{\nwtagstyle{}\subpageref{NW0-3ji0t0-1}}\RA{}
\LA{}Function declarations~{\nwtagstyle{}\subpageref{NW0-Xqfpb-1}}\RA{}
\LA{}Functions~{\nwtagstyle{}\subpageref{NW0-1byJZg-1}}\RA{}
\LA{}Main function~{\nwtagstyle{}\subpageref{NW0-32ejEQ-1}}\RA{}

\nwnotused{midRoot.c}\nwendcode{}\nwbegindocs{3}\nwdocspar
The \texttt{main} function is implemented first, as it
determines the rest.

\nwenddocs{}\nwbegincode{4}\sublabel{NW0-32ejEQ-1}\nwmargintag{{\nwtagstyle{}\subpageref{NW0-32ejEQ-1}}}\moddef{Main function~{\nwtagstyle{}\subpageref{NW0-32ejEQ-1}}}\endmoddef\nwstartdeflinemarkup\nwusesondefline{\\{NW0-5D8f4-1}}\nwenddeflinemarkup
int \nwlinkedidentc{main}{NW0-32ejEQ-1}(int argc, char **argv) \{\nwindexdefn{\nwixident{main}}{main}{NW0-32ejEQ-1}
  Args *args = getArgs(argc, argv);
  setProgName(argv[0]);
  \LA{}Deal with user interface~{\nwtagstyle{}\subpageref{NW0-3hJosa-1}}\RA{}
  \LA{}Deal with input files~{\nwtagstyle{}\subpageref{NW0-4EgRTZ-1}}\RA{}
  freeArgs(args);
  free(getProgName());
  return 0;
\}

\nwused{\\{NW0-5D8f4-1}}\nwidentdefs{\\{{\nwixident{main}}{main}}}\nwendcode{}\nwbegindocs{5}\nwdocspar
The functions \texttt{getArgs} and \texttt{freeArgs} are declared in
\texttt{interface.h}, and the functions \texttt{setProgName} and
\texttt{getProgName} in \texttt{eprintf.h}. Both of these headers and
their implementations are part of this package. The function
\texttt{free} is part of the standard library.

\nwenddocs{}\nwbegincode{6}\sublabel{NW0-3ji0t0-1}\nwmargintag{{\nwtagstyle{}\subpageref{NW0-3ji0t0-1}}}\moddef{Include headers~{\nwtagstyle{}\subpageref{NW0-3ji0t0-1}}}\endmoddef\nwstartdeflinemarkup\nwusesondefline{\\{NW0-5D8f4-1}}\nwprevnextdefs{\relax}{NW0-3ji0t0-2}\nwenddeflinemarkup
#include "interface.h"
#include "eprintf.h"
#include <stdlib.h>

\nwalsodefined{\\{NW0-3ji0t0-2}\\{NW0-3ji0t0-3}}\nwused{\\{NW0-5D8f4-1}}\nwendcode{}\nwbegindocs{7}\nwdocspar
The user might have requested the
program version, in which case a---rather modest---splash screen is
printed.

\nwenddocs{}\nwbegincode{8}\sublabel{NW0-3hJosa-1}\nwmargintag{{\nwtagstyle{}\subpageref{NW0-3hJosa-1}}}\moddef{Deal with user interface~{\nwtagstyle{}\subpageref{NW0-3hJosa-1}}}\endmoddef\nwstartdeflinemarkup\nwusesondefline{\\{NW0-32ejEQ-1}}\nwprevnextdefs{\relax}{NW0-3hJosa-2}\nwenddeflinemarkup
if (args->v)
  printSplash(args);

\nwalsodefined{\\{NW0-3hJosa-2}}\nwused{\\{NW0-32ejEQ-1}}\nwendcode{}\nwbegindocs{9}\nwdocspar
Alternatively, the user might request help, or an error has
occurred. In both cases a usage message is printed. Printing
usage or splash is declared in \texttt{interface.h}, which is already included.

\nwenddocs{}\nwbegincode{10}\sublabel{NW0-3hJosa-2}\nwmargintag{{\nwtagstyle{}\subpageref{NW0-3hJosa-2}}}\moddef{Deal with user interface~{\nwtagstyle{}\subpageref{NW0-3hJosa-1}}}\plusendmoddef\nwstartdeflinemarkup\nwusesondefline{\\{NW0-32ejEQ-1}}\nwprevnextdefs{NW0-3hJosa-1}{\relax}\nwenddeflinemarkup
if (args->h || args->err)
  printUsage();

\nwused{\\{NW0-32ejEQ-1}}\nwendcode{}\nwbegindocs{11}\nwdocspar
Input is dealt with in the standard UNIX manner: If the user lists no
files, \texttt{stdin} is the input stream; otherwise, each file in
turn becomes the input stream.

\nwenddocs{}\nwbegincode{12}\sublabel{NW0-4EgRTZ-1}\nwmargintag{{\nwtagstyle{}\subpageref{NW0-4EgRTZ-1}}}\moddef{Deal with input files~{\nwtagstyle{}\subpageref{NW0-4EgRTZ-1}}}\endmoddef\nwstartdeflinemarkup\nwusesondefline{\\{NW0-32ejEQ-1}}\nwenddeflinemarkup
if (args->nf == 0) \{
  FILE *fp = stdin;
  \nwlinkedidentc{scanFile}{NW0-1byJZg-1}(fp, args);
\} else \{
  for (int i = 0; i < args->nf; i++) \{
    FILE *fp = efopen(args->fi[i], "r");
    \nwlinkedidentc{scanFile}{NW0-1byJZg-1}(fp, args);
    fclose(fp);
  \}
\}

\nwused{\\{NW0-32ejEQ-1}}\nwidentuses{\\{{\nwixident{scanFile}}{scanFile}}}\nwindexuse{\nwixident{scanFile}}{scanFile}{NW0-4EgRTZ-1}\nwendcode{}\nwbegindocs{13}\nwdocspar
The function \texttt{scanFile} is declared.

\nwenddocs{}\nwbegincode{14}\sublabel{NW0-Xqfpb-1}\nwmargintag{{\nwtagstyle{}\subpageref{NW0-Xqfpb-1}}}\moddef{Function declarations~{\nwtagstyle{}\subpageref{NW0-Xqfpb-1}}}\endmoddef\nwstartdeflinemarkup\nwusesondefline{\\{NW0-5D8f4-1}}\nwprevnextdefs{\relax}{NW0-Xqfpb-2}\nwenddeflinemarkup
void \nwlinkedidentc{scanFile}{NW0-1byJZg-1}(FILE *fp, Args *args);

\nwalsodefined{\\{NW0-Xqfpb-2}\\{NW0-Xqfpb-3}\\{NW0-Xqfpb-4}}\nwused{\\{NW0-5D8f4-1}}\nwidentuses{\\{{\nwixident{scanFile}}{scanFile}}}\nwindexuse{\nwixident{scanFile}}{scanFile}{NW0-Xqfpb-1}\nwendcode{}\nwbegindocs{15}\nwdocspar
It retrieves each tree from a given file, roots it, and then prints it
in the Newick format favored by biologists. Joe Felsenstein, who is
one of its inventors, describes it on his web site:
\small
\begin{verbatim}
http://evolution.genetics.washington.edu/phylip/newicktree.html
\end{verbatim}
\normalsize

\nwenddocs{}\nwbegincode{16}\sublabel{NW0-1byJZg-1}\nwmargintag{{\nwtagstyle{}\subpageref{NW0-1byJZg-1}}}\moddef{Functions~{\nwtagstyle{}\subpageref{NW0-1byJZg-1}}}\endmoddef\nwstartdeflinemarkup\nwusesondefline{\\{NW0-5D8f4-1}}\nwprevnextdefs{\relax}{NW0-1byJZg-2}\nwenddeflinemarkup
void \nwlinkedidentc{scanFile}{NW0-1byJZg-1}(FILE *fp, Args *args) \{\nwindexdefn{\nwixident{scanFile}}{scanFile}{NW0-1byJZg-1}
  setBisonFile(fp);
  Node *r, *m;
  while ((r = parseTree()) != NULL) \{
    m = \nwlinkedidentc{midRoot}{NW0-Xqfpb-2}(r);
    printNewick(m);
    freeTree(m);
  \}
\}

\nwalsodefined{\\{NW0-1byJZg-2}\\{NW0-1byJZg-3}\\{NW0-1byJZg-4}}\nwused{\\{NW0-5D8f4-1}}\nwidentdefs{\\{{\nwixident{scanFile}}{scanFile}}}\nwidentuses{\\{{\nwixident{midRoot}}{midRoot}}}\nwindexuse{\nwixident{midRoot}}{midRoot}{NW0-1byJZg-1}\nwendcode{}\nwbegindocs{17}\nwdocspar
The functions \texttt{setBisonFile}, \texttt{parseTree},
\texttt{printNewick}, and \texttt{freeTree} are part of the
\texttt{tree} library distributed with this package.

\nwenddocs{}\nwbegincode{18}\sublabel{NW0-3ji0t0-2}\nwmargintag{{\nwtagstyle{}\subpageref{NW0-3ji0t0-2}}}\moddef{Include headers~{\nwtagstyle{}\subpageref{NW0-3ji0t0-1}}}\plusendmoddef\nwstartdeflinemarkup\nwusesondefline{\\{NW0-5D8f4-1}}\nwprevnextdefs{NW0-3ji0t0-1}{NW0-3ji0t0-3}\nwenddeflinemarkup
#include "tree.h"

\nwused{\\{NW0-5D8f4-1}}\nwendcode{}\nwbegindocs{19}\nwdocspar
The function \texttt{midRoot} is declared.

\nwenddocs{}\nwbegincode{20}\sublabel{NW0-Xqfpb-2}\nwmargintag{{\nwtagstyle{}\subpageref{NW0-Xqfpb-2}}}\moddef{Function declarations~{\nwtagstyle{}\subpageref{NW0-Xqfpb-1}}}\plusendmoddef\nwstartdeflinemarkup\nwusesondefline{\\{NW0-5D8f4-1}}\nwprevnextdefs{NW0-Xqfpb-1}{NW0-Xqfpb-3}\nwenddeflinemarkup
Node *\nwlinkedidentc{midRoot}{NW0-Xqfpb-2}(Node *r);\nwindexdefn{\nwixident{midRoot}}{midRoot}{NW0-Xqfpb-2}

\nwused{\\{NW0-5D8f4-1}}\nwidentdefs{\\{{\nwixident{midRoot}}{midRoot}}}\nwendcode{}\nwbegindocs{21}\nwdocspar
It first searches for the pair of leaves with the largest
distance. The new root is inserted midpoint between these leaves, and
then the tree is rearranged accordingly. At the end of the program we
free any memory allocated along the way is freed and the new root
returned.

\nwenddocs{}\nwbegincode{22}\sublabel{NW0-1byJZg-2}\nwmargintag{{\nwtagstyle{}\subpageref{NW0-1byJZg-2}}}\moddef{Functions~{\nwtagstyle{}\subpageref{NW0-1byJZg-1}}}\plusendmoddef\nwstartdeflinemarkup\nwusesondefline{\\{NW0-5D8f4-1}}\nwprevnextdefs{NW0-1byJZg-1}{NW0-1byJZg-3}\nwenddeflinemarkup
Node *\nwlinkedidentc{midRoot}{NW0-Xqfpb-2}(Node *r) \{\nwindexdefn{\nwixident{midRoot}}{midRoot}{NW0-1byJZg-2}
  \LA{}Find max. pair~{\nwtagstyle{}\subpageref{NW0-2LcRFk-1}}\RA{}
  \LA{}Insert new root~{\nwtagstyle{}\subpageref{NW0-13oY2G-1}}\RA{}
  \LA{}Rearrange tree~{\nwtagstyle{}\subpageref{NW0-3PQBB5-1}}\RA{}
  \LA{}Free memory~{\nwtagstyle{}\subpageref{NW0-Dn4oa-1}}\RA{}
  return r;
\}

\nwused{\\{NW0-5D8f4-1}}\nwidentdefs{\\{{\nwixident{midRoot}}{midRoot}}}\nwendcode{}\nwbegindocs{23}\nwdocspar
To find the pair of leaves furthest apart, $(\ell_i,\ell_j)$, for each
pair of leaves the distance to their lowes common ancestor,
$\mathrm{lca}$, is computed. These values reveal the leaves furthest
apart on the tree.
\nwenddocs{}\nwbegincode{24}\sublabel{NW0-2LcRFk-1}\nwmargintag{{\nwtagstyle{}\subpageref{NW0-2LcRFk-1}}}\moddef{Find max. pair~{\nwtagstyle{}\subpageref{NW0-2LcRFk-1}}}\endmoddef\nwstartdeflinemarkup\nwusesondefline{\\{NW0-1byJZg-2}}\nwenddeflinemarkup
int \nwlinkedidentc{n}{NW0-2LcRFk-1} = 0;\nwindexdefn{\nwixident{n}}{n}{NW0-2LcRFk-1}
Node **\nwlinkedidentc{leaves}{NW0-2LcRFk-1} = NULL;\nwindexdefn{\nwixident{leaves}}{leaves}{NW0-2LcRFk-1}
\nwlinkedidentc{leaves}{NW0-2LcRFk-1} = \nwlinkedidentc{getLeaves}{NW0-Xqfpb-3}(r, \nwlinkedidentc{leaves}{NW0-2LcRFk-1}, &\nwlinkedidentc{n}{NW0-2LcRFk-1});
\LA{}Compute $d(\ell_i,\mathrm{lca})$~{\nwtagstyle{}\subpageref{NW0-3wCMIx-1}}\RA{}
\LA{}Look for maximum distance~{\nwtagstyle{}\subpageref{NW0-1NLudF-1}}\RA{}
\nwused{\\{NW0-1byJZg-2}}\nwidentdefs{\\{{\nwixident{leaves}}{leaves}}\\{{\nwixident{n}}{n}}}\nwidentuses{\\{{\nwixident{getLeaves}}{getLeaves}}}\nwindexuse{\nwixident{getLeaves}}{getLeaves}{NW0-2LcRFk-1}\nwendcode{}\nwbegindocs{25}\nwdocspar
The leaves are freed at the end of the run.
\nwenddocs{}\nwbegincode{26}\sublabel{NW0-Dn4oa-1}\nwmargintag{{\nwtagstyle{}\subpageref{NW0-Dn4oa-1}}}\moddef{Free memory~{\nwtagstyle{}\subpageref{NW0-Dn4oa-1}}}\endmoddef\nwstartdeflinemarkup\nwusesondefline{\\{NW0-1byJZg-2}}\nwprevnextdefs{\relax}{NW0-Dn4oa-2}\nwenddeflinemarkup
free(\nwlinkedidentc{leaves}{NW0-2LcRFk-1});
\nwalsodefined{\\{NW0-Dn4oa-2}}\nwused{\\{NW0-1byJZg-2}}\nwidentuses{\\{{\nwixident{leaves}}{leaves}}}\nwindexuse{\nwixident{leaves}}{leaves}{NW0-Dn4oa-1}\nwendcode{}\nwbegindocs{27}\nwdocspar
The function \texttt{getLeaves} is declared
\nwenddocs{}\nwbegincode{28}\sublabel{NW0-Xqfpb-3}\nwmargintag{{\nwtagstyle{}\subpageref{NW0-Xqfpb-3}}}\moddef{Function declarations~{\nwtagstyle{}\subpageref{NW0-Xqfpb-1}}}\plusendmoddef\nwstartdeflinemarkup\nwusesondefline{\\{NW0-5D8f4-1}}\nwprevnextdefs{NW0-Xqfpb-2}{NW0-Xqfpb-4}\nwenddeflinemarkup
Node **\nwlinkedidentc{getLeaves}{NW0-Xqfpb-3}(Node *r, Node **\nwlinkedidentc{leaves}{NW0-2LcRFk-1}, int *\nwlinkedidentc{n}{NW0-2LcRFk-1});\nwindexdefn{\nwixident{getLeaves}}{getLeaves}{NW0-Xqfpb-3}
\nwused{\\{NW0-5D8f4-1}}\nwidentdefs{\\{{\nwixident{getLeaves}}{getLeaves}}}\nwidentuses{\\{{\nwixident{leaves}}{leaves}}\\{{\nwixident{n}}{n}}}\nwindexuse{\nwixident{leaves}}{leaves}{NW0-Xqfpb-3}\nwindexuse{\nwixident{n}}{n}{NW0-Xqfpb-3}\nwendcode{}\nwbegindocs{29}\nwdocspar
and implemented by direct recursion.
\nwenddocs{}\nwbegincode{30}\sublabel{NW0-1byJZg-3}\nwmargintag{{\nwtagstyle{}\subpageref{NW0-1byJZg-3}}}\moddef{Functions~{\nwtagstyle{}\subpageref{NW0-1byJZg-1}}}\plusendmoddef\nwstartdeflinemarkup\nwusesondefline{\\{NW0-5D8f4-1}}\nwprevnextdefs{NW0-1byJZg-2}{NW0-1byJZg-4}\nwenddeflinemarkup
Node **\nwlinkedidentc{getLeaves}{NW0-Xqfpb-3}(Node *r, Node **\nwlinkedidentc{leaves}{NW0-2LcRFk-1}, int *\nwlinkedidentc{n}{NW0-2LcRFk-1}) \{\nwindexdefn{\nwixident{getLeaves}}{getLeaves}{NW0-1byJZg-3}
  if (r != NULL) \{
    \nwlinkedidentc{leaves}{NW0-2LcRFk-1} = \nwlinkedidentc{getLeaves}{NW0-Xqfpb-3}(r->left, \nwlinkedidentc{leaves}{NW0-2LcRFk-1}, \nwlinkedidentc{n}{NW0-2LcRFk-1});
    \nwlinkedidentc{leaves}{NW0-2LcRFk-1} = \nwlinkedidentc{getLeaves}{NW0-Xqfpb-3}(r->right, \nwlinkedidentc{leaves}{NW0-2LcRFk-1}, \nwlinkedidentc{n}{NW0-2LcRFk-1});
    if (r->left == NULL) \{
        *\nwlinkedidentc{n}{NW0-2LcRFk-1} = *\nwlinkedidentc{n}{NW0-2LcRFk-1} + 1;
        \nwlinkedidentc{leaves}{NW0-2LcRFk-1} = (Node **)erealloc(\nwlinkedidentc{leaves}{NW0-2LcRFk-1}, *\nwlinkedidentc{n}{NW0-2LcRFk-1} * sizeof(Node *));
        \nwlinkedidentc{leaves}{NW0-2LcRFk-1}[*\nwlinkedidentc{n}{NW0-2LcRFk-1} - 1] = r;
    \}
  \}
  return \nwlinkedidentc{leaves}{NW0-2LcRFk-1};
\}
\nwused{\\{NW0-5D8f4-1}}\nwidentdefs{\\{{\nwixident{getLeaves}}{getLeaves}}}\nwidentuses{\\{{\nwixident{leaves}}{leaves}}\\{{\nwixident{n}}{n}}}\nwindexuse{\nwixident{leaves}}{leaves}{NW0-1byJZg-3}\nwindexuse{\nwixident{n}}{n}{NW0-1byJZg-3}\nwendcode{}\nwbegindocs{31}\nwdocspar
Having obtained the leaves, their distances to the lowes common
ancestor (\emph{lca}) are computed for all pairs.
\nwenddocs{}\nwbegincode{32}\sublabel{NW0-3wCMIx-1}\nwmargintag{{\nwtagstyle{}\subpageref{NW0-3wCMIx-1}}}\moddef{Compute $d(\ell_i,\mathrm{lca})$~{\nwtagstyle{}\subpageref{NW0-3wCMIx-1}}}\endmoddef\nwstartdeflinemarkup\nwusesondefline{\\{NW0-2LcRFk-1}}\nwenddeflinemarkup
double **\nwlinkedidentc{dlca}{NW0-3wCMIx-1} = (double **)emalloc(\nwlinkedidentc{n}{NW0-2LcRFk-1} * sizeof(double *));\nwindexdefn{\nwixident{dlca}}{dlca}{NW0-3wCMIx-1}
for (int i = 0; i < \nwlinkedidentc{n}{NW0-2LcRFk-1}; i++)
  \nwlinkedidentc{dlca}{NW0-3wCMIx-1}[i] = (double *)emalloc(\nwlinkedidentc{n}{NW0-2LcRFk-1} * sizeof(double));
for (int i = 0; i < \nwlinkedidentc{n}{NW0-2LcRFk-1} - 1; i++)
  for (int j = i; j < \nwlinkedidentc{n}{NW0-2LcRFk-1}; j++) \{
    Node *la = lca(\nwlinkedidentc{leaves}{NW0-2LcRFk-1}[i], \nwlinkedidentc{leaves}{NW0-2LcRFk-1}[j]);
    \nwlinkedidentc{dlca}{NW0-3wCMIx-1}[i][j] = ancDist(\nwlinkedidentc{leaves}{NW0-2LcRFk-1}[i], la);
    \nwlinkedidentc{dlca}{NW0-3wCMIx-1}[j][i] = ancDist(\nwlinkedidentc{leaves}{NW0-2LcRFk-1}[j], la);
  \}
\nwused{\\{NW0-2LcRFk-1}}\nwidentdefs{\\{{\nwixident{dlca}}{dlca}}}\nwidentuses{\\{{\nwixident{leaves}}{leaves}}\\{{\nwixident{n}}{n}}}\nwindexuse{\nwixident{leaves}}{leaves}{NW0-3wCMIx-1}\nwindexuse{\nwixident{n}}{n}{NW0-3wCMIx-1}\nwendcode{}\nwbegindocs{33}\nwdocspar
The memory for $d(\ell_i, \mathrm{lca})$ is freed at the end.
\nwenddocs{}\nwbegincode{34}\sublabel{NW0-Dn4oa-2}\nwmargintag{{\nwtagstyle{}\subpageref{NW0-Dn4oa-2}}}\moddef{Free memory~{\nwtagstyle{}\subpageref{NW0-Dn4oa-1}}}\plusendmoddef\nwstartdeflinemarkup\nwusesondefline{\\{NW0-1byJZg-2}}\nwprevnextdefs{NW0-Dn4oa-1}{\relax}\nwenddeflinemarkup
for (int i = 0; i < \nwlinkedidentc{n}{NW0-2LcRFk-1}; i++)
  free(\nwlinkedidentc{dlca}{NW0-3wCMIx-1}[i]);
free(\nwlinkedidentc{dlca}{NW0-3wCMIx-1});
\nwused{\\{NW0-1byJZg-2}}\nwidentuses{\\{{\nwixident{dlca}}{dlca}}\\{{\nwixident{n}}{n}}}\nwindexuse{\nwixident{dlca}}{dlca}{NW0-Dn4oa-2}\nwindexuse{\nwixident{n}}{n}{NW0-Dn4oa-2}\nwendcode{}\nwbegindocs{35}\nwdocspar
Now everything is ready for finding the pair of leaves,
$(\ell_i,\ell_j)$, with the maximum distance, $d_{\rm m}$.
\nwenddocs{}\nwbegincode{36}\sublabel{NW0-1NLudF-1}\nwmargintag{{\nwtagstyle{}\subpageref{NW0-1NLudF-1}}}\moddef{Look for maximum distance~{\nwtagstyle{}\subpageref{NW0-1NLudF-1}}}\endmoddef\nwstartdeflinemarkup\nwusesondefline{\\{NW0-2LcRFk-1}}\nwenddeflinemarkup
double \nwlinkedidentc{d}{NW0-1NLudF-1}, \nwlinkedidentc{dm}{NW0-1NLudF-1} = 0;\nwindexdefn{\nwixident{d}}{d}{NW0-1NLudF-1}\nwindexdefn{\nwixident{dm}}{dm}{NW0-1NLudF-1}
int \nwlinkedidentc{im}{NW0-1NLudF-1} = 0, jm = 0;\nwindexdefn{\nwixident{im}}{im}{NW0-1NLudF-1}
for (int i = 0; i < \nwlinkedidentc{n}{NW0-2LcRFk-1} - 1; i++) \{
  for (int j = i + 1; j < \nwlinkedidentc{n}{NW0-2LcRFk-1}; j++) \{
    \nwlinkedidentc{d}{NW0-1NLudF-1} = \nwlinkedidentc{dlca}{NW0-3wCMIx-1}[i][j] + \nwlinkedidentc{dlca}{NW0-3wCMIx-1}[j][i];
    if (\nwlinkedidentc{d}{NW0-1NLudF-1} > \nwlinkedidentc{dm}{NW0-1NLudF-1}) \{
        \nwlinkedidentc{dm}{NW0-1NLudF-1} = \nwlinkedidentc{d}{NW0-1NLudF-1};
        \nwlinkedidentc{im}{NW0-1NLudF-1} = i;
        jm = j;
    \}
  \}
\}

\nwused{\\{NW0-2LcRFk-1}}\nwidentdefs{\\{{\nwixident{d}}{d}}\\{{\nwixident{dm}}{dm}}\\{{\nwixident{im}}{im}}}\nwidentuses{\\{{\nwixident{dlca}}{dlca}}\\{{\nwixident{n}}{n}}}\nwindexuse{\nwixident{dlca}}{dlca}{NW0-1NLudF-1}\nwindexuse{\nwixident{n}}{n}{NW0-1NLudF-1}\nwendcode{}\nwbegindocs{37}\nwdocspar
To insert the new root in the middle between $\ell_i$ and $\ell_j$,
the relevant edge is located and split. Figure~\ref{fig:ur}A shows a
version of Figure~\ref{fig:phy}A where the branch lengths have been
abstracted and the internal nodes are labeled. Internally, such
``unrooted trees'' are in fact rooted, which is made explicit in
Figure~\ref{fig:ur}B. To re-root along the edge $(t_1,n_2)$, it is
located and a new node, $n_5$, added (Figure~\ref{fig:ur}C).
\nwenddocs{}\nwbegincode{38}\sublabel{NW0-13oY2G-1}\nwmargintag{{\nwtagstyle{}\subpageref{NW0-13oY2G-1}}}\moddef{Insert new root~{\nwtagstyle{}\subpageref{NW0-13oY2G-1}}}\endmoddef\nwstartdeflinemarkup\nwusesondefline{\\{NW0-1byJZg-2}}\nwenddeflinemarkup
\LA{}Find edge to split~{\nwtagstyle{}\subpageref{NW0-44Su8C-1}}\RA{}
\LA{}Add new node~{\nwtagstyle{}\subpageref{NW0-1nnmsk-1}}\RA{}
\nwused{\\{NW0-1byJZg-2}}\nwendcode{}\nwbegindocs{39}\nwdocspar
To find the edge to split, a climb towards the root is started from
either $\ell_i$ or $\ell_j$, whichever has the greater distance to the
lowest common ancestor. The climb continues until the sum of edge
lengths exceeds $d_{\rm m} / 2$.
\nwenddocs{}\nwbegincode{40}\sublabel{NW0-44Su8C-1}\nwmargintag{{\nwtagstyle{}\subpageref{NW0-44Su8C-1}}}\moddef{Find edge to split~{\nwtagstyle{}\subpageref{NW0-44Su8C-1}}}\endmoddef\nwstartdeflinemarkup\nwusesondefline{\\{NW0-13oY2G-1}}\nwenddeflinemarkup
Node *\nwlinkedidentc{c}{NW0-44Su8C-1}, *\nwlinkedidentc{p}{NW0-44Su8C-1};  /* child, parent */\nwindexdefn{\nwixident{c}}{c}{NW0-44Su8C-1}\nwindexdefn{\nwixident{p}}{p}{NW0-44Su8C-1}
double \nwlinkedidentc{dh}{NW0-44Su8C-1} = \nwlinkedidentc{dm}{NW0-1NLudF-1} / 2.;\nwindexdefn{\nwixident{dh}}{dh}{NW0-44Su8C-1}
if (\nwlinkedidentc{dlca}{NW0-3wCMIx-1}[\nwlinkedidentc{im}{NW0-1NLudF-1}][jm] > \nwlinkedidentc{dh}{NW0-44Su8C-1})
  \nwlinkedidentc{c}{NW0-44Su8C-1} = \nwlinkedidentc{leaves}{NW0-2LcRFk-1}[\nwlinkedidentc{im}{NW0-1NLudF-1}];
else
  \nwlinkedidentc{c}{NW0-44Su8C-1} = \nwlinkedidentc{leaves}{NW0-2LcRFk-1}[jm];
double \nwlinkedidentc{s}{NW0-44Su8C-1} = \nwlinkedidentc{c}{NW0-44Su8C-1}->dist;\nwindexdefn{\nwixident{s}}{s}{NW0-44Su8C-1}
while(\nwlinkedidentc{c}{NW0-44Su8C-1}->parent && \nwlinkedidentc{s}{NW0-44Su8C-1} < \nwlinkedidentc{dh}{NW0-44Su8C-1}) \{
  \nwlinkedidentc{c}{NW0-44Su8C-1} = \nwlinkedidentc{c}{NW0-44Su8C-1}->parent;
  \nwlinkedidentc{s}{NW0-44Su8C-1} += \nwlinkedidentc{c}{NW0-44Su8C-1}->dist;
\}
\nwlinkedidentc{p}{NW0-44Su8C-1} = \nwlinkedidentc{c}{NW0-44Su8C-1}->parent;
\nwused{\\{NW0-13oY2G-1}}\nwidentdefs{\\{{\nwixident{c}}{c}}\\{{\nwixident{dh}}{dh}}\\{{\nwixident{p}}{p}}\\{{\nwixident{s}}{s}}}\nwidentuses{\\{{\nwixident{dlca}}{dlca}}\\{{\nwixident{dm}}{dm}}\\{{\nwixident{im}}{im}}\\{{\nwixident{leaves}}{leaves}}}\nwindexuse{\nwixident{dlca}}{dlca}{NW0-44Su8C-1}\nwindexuse{\nwixident{dm}}{dm}{NW0-44Su8C-1}\nwindexuse{\nwixident{im}}{im}{NW0-44Su8C-1}\nwindexuse{\nwixident{leaves}}{leaves}{NW0-44Su8C-1}\nwendcode{}\nwbegindocs{41}\nwdocspar
Now the new root node, $n_5$, is created, the edge $(t_1, n_2)$ is
split, and $n_5$ spliced in (Figure~\ref{fig:ur}C). The splicing is
done in three steps: First, $n_5$ is added to the children of $n_2$,
then $t_1$ is removed from the children of $n_2$, and $t_1$ added to
the children of $n_5$. Finally, the distances to and from $n_5$ are
adjusted.
\nwenddocs{}\nwbegincode{42}\sublabel{NW0-1nnmsk-1}\nwmargintag{{\nwtagstyle{}\subpageref{NW0-1nnmsk-1}}}\moddef{Add new node~{\nwtagstyle{}\subpageref{NW0-1nnmsk-1}}}\endmoddef\nwstartdeflinemarkup\nwusesondefline{\\{NW0-13oY2G-1}}\nwenddeflinemarkup
Node *\nwlinkedidentc{R}{NW0-1nnmsk-1} = newNode(); /* new root */\nwindexdefn{\nwixident{R}}{R}{NW0-1nnmsk-1}
addChild(\nwlinkedidentc{p}{NW0-44Su8C-1}, \nwlinkedidentc{R}{NW0-1nnmsk-1});
rmChild(\nwlinkedidentc{p}{NW0-44Su8C-1}, \nwlinkedidentc{c}{NW0-44Su8C-1});
addChild(\nwlinkedidentc{R}{NW0-1nnmsk-1}, \nwlinkedidentc{c}{NW0-44Su8C-1});
\LA{}Adjust distances~{\nwtagstyle{}\subpageref{NW0-24elOi-1}}\RA{}
\nwused{\\{NW0-13oY2G-1}}\nwidentdefs{\\{{\nwixident{R}}{R}}}\nwidentuses{\\{{\nwixident{c}}{c}}\\{{\nwixident{p}}{p}}}\nwindexuse{\nwixident{c}}{c}{NW0-1nnmsk-1}\nwindexuse{\nwixident{p}}{p}{NW0-1nnmsk-1}\nwendcode{}\nwbegindocs{43}\nwdocspar
The functions \texttt{addNode} and \texttt{rmChild} are already part
of the \texttt{tree} library.  As to the distances, let $s$ be the
length of the path $(t_1,n_2)$, and call the desired lengths
$x_1=d(t_1,n_5)$ and $x_2=d(n_5,n_2)$. Then $x_2 = s - d_{\rm m} /2$
and $x_1 = d(t_1,n_2) - x_2$.
\nwenddocs{}\nwbegincode{44}\sublabel{NW0-24elOi-1}\nwmargintag{{\nwtagstyle{}\subpageref{NW0-24elOi-1}}}\moddef{Adjust distances~{\nwtagstyle{}\subpageref{NW0-24elOi-1}}}\endmoddef\nwstartdeflinemarkup\nwusesondefline{\\{NW0-1nnmsk-1}}\nwenddeflinemarkup
float \nwlinkedidentc{x2}{NW0-24elOi-1} = \nwlinkedidentc{s}{NW0-44Su8C-1} - \nwlinkedidentc{dm}{NW0-1NLudF-1} / 2.;\nwindexdefn{\nwixident{x2}}{x2}{NW0-24elOi-1}
float \nwlinkedidentc{x1}{NW0-24elOi-1} = \nwlinkedidentc{c}{NW0-44Su8C-1}->dist - \nwlinkedidentc{x2}{NW0-24elOi-1};\nwindexdefn{\nwixident{x1}}{x1}{NW0-24elOi-1}
\nwlinkedidentc{c}{NW0-44Su8C-1}->dist = \nwlinkedidentc{x1}{NW0-24elOi-1};
\nwlinkedidentc{R}{NW0-1nnmsk-1}->dist = \nwlinkedidentc{x2}{NW0-24elOi-1};
\nwused{\\{NW0-1nnmsk-1}}\nwidentdefs{\\{{\nwixident{x1}}{x1}}\\{{\nwixident{x2}}{x2}}}\nwidentuses{\\{{\nwixident{c}}{c}}\\{{\nwixident{dm}}{dm}}\\{{\nwixident{R}}{R}}\\{{\nwixident{s}}{s}}}\nwindexuse{\nwixident{c}}{c}{NW0-24elOi-1}\nwindexuse{\nwixident{dm}}{dm}{NW0-24elOi-1}\nwindexuse{\nwixident{R}}{R}{NW0-24elOi-1}\nwindexuse{\nwixident{s}}{s}{NW0-24elOi-1}\nwendcode{}\nwbegindocs{45}\nwdocspar
The last step is to rearrange the tree such that $n_5$ becomes the new
root. Donald Knuth described the process for this as picking up the
tree by $n_5$ in Figure~\ref{fig:ur}C, and shaking it to get
Figure~\ref{fig:ur}D~\cite[p. 371]{knu97:ar1}. More formally, this is
done by climbing from $n_5$ to the old root, $n_1$, and at every step
converting the parent to a child node and adjusting the distance
attribute accordingly. For this purpose, the new function
\texttt{parentToChild} is applied to $n_5$, before making it the new
root, which moves the algorithm from Figure~\ref{fig:ur}C to
Figure~\ref{fig:ur}D.
\begin{figure}
\psfrag{T1}{\Huge$t_1$}
\psfrag{T2}{\Huge$t_2$}
\psfrag{T3}{\Huge$t_3$}
\psfrag{T4}{\Huge$t_4$}
\psfrag{T5}{\Huge$t_5$}
\psfrag{T6}{\Huge$t_6$}

\psfrag{N2}{\Huge$n_1$}
\psfrag{N4}{\Huge$n_2$}
\psfrag{N6}{\Huge$n_3$}
\psfrag{N9}{\Huge$n_4$}
\psfrag{N11}{\Huge$n_5$}
\psfrag{<<}{\hspace{-0.5cm}\rotatebox{45}{\Huge$||$}}

  \begin{center}
    \begin{tabular}{cc}
          \textbf{A} & \textbf{B}\\
          \scalebox{0.4}{\includegraphics{smallU}} & \scalebox{0.4}{\includegraphics{smallR}}\\
          \textbf{C} & \textbf{D}\\
          \scalebox{0.4}{\includegraphics{smallR2}} & \scalebox{0.4}{\includegraphics{smallR3}}
        \end{tabular}
  \end{center}
  \caption{Rooting a tree. (\textbf{A}) is drawn in the unrooted,
    radial layout often used in biology, even though internally it is
    rooted on $n_1$. This rooting is made explicit in (\textbf{B}),
    where we wish to reroot the tree on edge $(t_1,n_2)$ marked by
    $||$.  In (\textbf{C}) the future new root $n_5$ is added to the
    tree by splitting $(t_1,n_2)$. (\textbf{D}) is the newly rooted
    tree. It is obtained by picking up (\textbf{C}) at $n_5$ and
    shaking it~\cite[p. 373]{knu97:ar1}.}\label{fig:ur}
\end{figure}
\nwenddocs{}\nwbegincode{46}\sublabel{NW0-3PQBB5-1}\nwmargintag{{\nwtagstyle{}\subpageref{NW0-3PQBB5-1}}}\moddef{Rearrange tree~{\nwtagstyle{}\subpageref{NW0-3PQBB5-1}}}\endmoddef\nwstartdeflinemarkup\nwusesondefline{\\{NW0-1byJZg-2}}\nwenddeflinemarkup
parentToChild(\nwlinkedidentc{R}{NW0-1nnmsk-1});
r = \nwlinkedidentc{R}{NW0-1nnmsk-1};
\nwused{\\{NW0-1byJZg-2}}\nwidentuses{\\{{\nwixident{R}}{R}}}\nwindexuse{\nwixident{R}}{R}{NW0-3PQBB5-1}\nwendcode{}\nwbegindocs{47}\nwdocspar
The function \texttt{parentToChild} is declared
\nwenddocs{}\nwbegincode{48}\sublabel{NW0-Xqfpb-4}\nwmargintag{{\nwtagstyle{}\subpageref{NW0-Xqfpb-4}}}\moddef{Function declarations~{\nwtagstyle{}\subpageref{NW0-Xqfpb-1}}}\plusendmoddef\nwstartdeflinemarkup\nwusesondefline{\\{NW0-5D8f4-1}}\nwprevnextdefs{NW0-Xqfpb-3}{\relax}\nwenddeflinemarkup
void parentToChild(Node *\nwlinkedidentc{n}{NW0-2LcRFk-1});
\nwused{\\{NW0-5D8f4-1}}\nwidentuses{\\{{\nwixident{n}}{n}}}\nwindexuse{\nwixident{n}}{n}{NW0-Xqfpb-4}\nwendcode{}\nwbegindocs{49}\nwdocspar
and then implemented. 
\nwenddocs{}\nwbegincode{50}\sublabel{NW0-1byJZg-4}\nwmargintag{{\nwtagstyle{}\subpageref{NW0-1byJZg-4}}}\moddef{Functions~{\nwtagstyle{}\subpageref{NW0-1byJZg-1}}}\plusendmoddef\nwstartdeflinemarkup\nwusesondefline{\\{NW0-5D8f4-1}}\nwprevnextdefs{NW0-1byJZg-3}{\relax}\nwenddeflinemarkup
  void parentToChild(Node *\nwlinkedidentc{n}{NW0-2LcRFk-1}) \{
    if(\nwlinkedidentc{n}{NW0-2LcRFk-1}->parent->parent != NULL)
        parentToChild(\nwlinkedidentc{n}{NW0-2LcRFk-1}->parent);
    Node *\nwlinkedidentc{p}{NW0-44Su8C-1} = \nwlinkedidentc{n}{NW0-2LcRFk-1}->parent;
    rmChild(\nwlinkedidentc{p}{NW0-44Su8C-1}, \nwlinkedidentc{n}{NW0-2LcRFk-1});
    addChild(\nwlinkedidentc{n}{NW0-2LcRFk-1}, \nwlinkedidentc{p}{NW0-44Su8C-1});
    \nwlinkedidentc{p}{NW0-44Su8C-1}->dist = \nwlinkedidentc{n}{NW0-2LcRFk-1}->dist;
  \}
\nwused{\\{NW0-5D8f4-1}}\nwidentuses{\\{{\nwixident{n}}{n}}\\{{\nwixident{p}}{p}}}\nwindexuse{\nwixident{n}}{n}{NW0-1byJZg-4}\nwindexuse{\nwixident{p}}{p}{NW0-1byJZg-4}\nwendcode{}\nwbegindocs{51}\nwdocspar
For \texttt{assert} a new header of the standard library is included.
\nwenddocs{}\nwbegincode{52}\sublabel{NW0-3ji0t0-3}\nwmargintag{{\nwtagstyle{}\subpageref{NW0-3ji0t0-3}}}\moddef{Include headers~{\nwtagstyle{}\subpageref{NW0-3ji0t0-1}}}\plusendmoddef\nwstartdeflinemarkup\nwusesondefline{\\{NW0-5D8f4-1}}\nwprevnextdefs{NW0-3ji0t0-2}{\relax}\nwenddeflinemarkup
#include <assert.h>
\nwused{\\{NW0-5D8f4-1}}\nwendcode{}\nwbegindocs{53}\nwdocspar
This was the final piece in the puzzle, midpoint rooting can now be
applied in pipelines like
\begin{verbatim}
andi foo.fasta | clustDist | midRoot | new2view
\end{verbatim}
where
\texttt{andi}\footnote{\texttt{https://github.com/evolbioinf/andi}}
computes the pairwise distances between the sequences in
\texttt{foo.fasta}~\cite{hau15:and},
\texttt{clustDist}\footnote{\texttt{https://github.com/evolbioinf/clustDist}}
transforms them to a neighbor joining tree, \texttt{midRoot} midpoint
roots it, and
\texttt{new2view}\footnote{\texttt{https://github.com/evolbioinf/new2view}}
finally draws the tree to the screen.
\nwenddocs{}\nwbegindocs{54}\nwdocspar
\section{List of code chunks}
\nowebchunks
\section{Index}
\nowebindex
\nwenddocs{}

\nwixlogsorted{c}{{Add new node}{NW0-1nnmsk-1}{\nwixu{NW0-13oY2G-1}\nwixd{NW0-1nnmsk-1}}}%
\nwixlogsorted{c}{{Adjust distances}{NW0-24elOi-1}{\nwixu{NW0-1nnmsk-1}\nwixd{NW0-24elOi-1}}}%
\nwixlogsorted{c}{{Compute $d(\ell_i,\mathrm{lca})$}{NW0-3wCMIx-1}{\nwixu{NW0-2LcRFk-1}\nwixd{NW0-3wCMIx-1}}}%
\nwixlogsorted{c}{{Deal with input files}{NW0-4EgRTZ-1}{\nwixu{NW0-32ejEQ-1}\nwixd{NW0-4EgRTZ-1}}}%
\nwixlogsorted{c}{{Deal with user interface}{NW0-3hJosa-1}{\nwixu{NW0-32ejEQ-1}\nwixd{NW0-3hJosa-1}\nwixd{NW0-3hJosa-2}}}%
\nwixlogsorted{c}{{Find edge to split}{NW0-44Su8C-1}{\nwixu{NW0-13oY2G-1}\nwixd{NW0-44Su8C-1}}}%
\nwixlogsorted{c}{{Find max. pair}{NW0-2LcRFk-1}{\nwixu{NW0-1byJZg-2}\nwixd{NW0-2LcRFk-1}}}%
\nwixlogsorted{c}{{Free memory}{NW0-Dn4oa-1}{\nwixu{NW0-1byJZg-2}\nwixd{NW0-Dn4oa-1}\nwixd{NW0-Dn4oa-2}}}%
\nwixlogsorted{c}{{Function declarations}{NW0-Xqfpb-1}{\nwixu{NW0-5D8f4-1}\nwixd{NW0-Xqfpb-1}\nwixd{NW0-Xqfpb-2}\nwixd{NW0-Xqfpb-3}\nwixd{NW0-Xqfpb-4}}}%
\nwixlogsorted{c}{{Functions}{NW0-1byJZg-1}{\nwixu{NW0-5D8f4-1}\nwixd{NW0-1byJZg-1}\nwixd{NW0-1byJZg-2}\nwixd{NW0-1byJZg-3}\nwixd{NW0-1byJZg-4}}}%
\nwixlogsorted{c}{{Include headers}{NW0-3ji0t0-1}{\nwixu{NW0-5D8f4-1}\nwixd{NW0-3ji0t0-1}\nwixd{NW0-3ji0t0-2}\nwixd{NW0-3ji0t0-3}}}%
\nwixlogsorted{c}{{Insert new root}{NW0-13oY2G-1}{\nwixu{NW0-1byJZg-2}\nwixd{NW0-13oY2G-1}}}%
\nwixlogsorted{c}{{Look for maximum distance}{NW0-1NLudF-1}{\nwixu{NW0-2LcRFk-1}\nwixd{NW0-1NLudF-1}}}%
\nwixlogsorted{c}{{Main function}{NW0-32ejEQ-1}{\nwixu{NW0-5D8f4-1}\nwixd{NW0-32ejEQ-1}}}%
\nwixlogsorted{c}{{midRoot.c}{NW0-5D8f4-1}{\nwixd{NW0-5D8f4-1}}}%
\nwixlogsorted{c}{{Rearrange tree}{NW0-3PQBB5-1}{\nwixu{NW0-1byJZg-2}\nwixd{NW0-3PQBB5-1}}}%
\nwixlogsorted{i}{{\nwixident{c}}{c}}%
\nwixlogsorted{i}{{\nwixident{d}}{d}}%
\nwixlogsorted{i}{{\nwixident{dh}}{dh}}%
\nwixlogsorted{i}{{\nwixident{dlca}}{dlca}}%
\nwixlogsorted{i}{{\nwixident{dm}}{dm}}%
\nwixlogsorted{i}{{\nwixident{getLeaves}}{getLeaves}}%
\nwixlogsorted{i}{{\nwixident{im}}{im}}%
\nwixlogsorted{i}{{\nwixident{leaves}}{leaves}}%
\nwixlogsorted{i}{{\nwixident{main}}{main}}%
\nwixlogsorted{i}{{\nwixident{midRoot}}{midRoot}}%
\nwixlogsorted{i}{{\nwixident{n}}{n}}%
\nwixlogsorted{i}{{\nwixident{p}}{p}}%
\nwixlogsorted{i}{{\nwixident{R}}{R}}%
\nwixlogsorted{i}{{\nwixident{s}}{s}}%
\nwixlogsorted{i}{{\nwixident{scanFile}}{scanFile}}%
\nwixlogsorted{i}{{\nwixident{x1}}{x1}}%
\nwixlogsorted{i}{{\nwixident{x2}}{x2}}%

